While our approach focused on a single simulation of the ecovisor, our use of
Mosaik demonstrates its effectiveness for large-scale smart grid simulations. In
future research, we suggest interconnecting multiple ecovisor systems to share
resources and further reduce carbon emissions. This network could be distributed
across different geographic regions, as carbon intensity varies depending on
location. By incorporating carbon information services like Electricity Maps
\cite{electricity_maps}, this could enable carbon-efficiency optimizations such
as Let’s Wait Awhile \cite{wiesner2021} or Cucumber \cite{wiesner2022} from
Wiesner et al. This would enhance the potential for carbon reduction at a larger
scale. This work also does not focus on monitoring and limiting the power
consumption of applications in the loop, due to the specific requirements of the
implementation. Therefore, a potential future direction could involve
incorporating an interface that enables various power measurement and control
techniques for different application hosts, such as containers, virtual
machines, or physical hardware systems.
