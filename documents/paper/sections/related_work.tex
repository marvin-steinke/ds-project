In research and development processes, the utilization of physical testbeds is a
widespread practice \cite{cintuglu2017, mambretti2015}. They serve as a medium
for testing and evaluating new technologies, systems, and products prior to
their market launch or further development. However, some scenarios and
conditions may not be feasible or safe to recreate in a physical testbed.
Simulations can provide a controlled and cost-effective environment for testing
such scenarios and conditions \cite{mansouri2020}. When specific physical or
software components require a certain degree of realism though, simulations or
co-simulations with SIL or HIL methodologies can effectively address these
limitations.

For instance, Beilharz et al. \cite{beilharz2021}, introduce Marvis, a framework
that provides a comprehensive staging environment for testing distributed IoT
applications. Marvis orchestrates hybrid testbeds, co-simulated domain
environments, and a central network simulation to create a representative
operating environment for the system. However, Marvis does not provide a
virtualized energy system, which is crucial to meet the requirements of our
problem statement.

Hagenmeyer et al. \cite{hagenmeyer2016} investigate the interplay of different
forms of energy on various value chains in Energy Lab 2.0. The focus is on
finding novel concepts to stabilize the volatile energy supply of renewables
through the use of storage systems and the application of information and
communication technology tools and algorithms. The smart energy system
simulation and control center is a key element of Energy Lab 2.0 and consists of
three parts: a power-hardware-in-the-loop experimental field, an energy grid
simulation and analysis laboratory, and a control, monitoring, and visualization
center. While this smart energy system simulation is similar to the simulated
ecovisor in our approach, the control center is the only entity with
software-based control over this energy system. The ecovisor infrastructure,
however, provides multiple applications with the ability to manage their energy
supply themselves which is an essential factor for our approach.

In conclusion, to the best of our knowledge, there is no current approach that
can simulate an energy system virtualizer with software-based control,
comparable to the ecovisor, with SIL capabilities.
