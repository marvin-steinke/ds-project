%TODO add some broader cluster refs to introduce
The Software- (SIL) or Hardware-In-The-Loop (HIL) methodology is commonly
utilized during the development of intricate systems as a means of conducting
repeatable and manageable component testing. For instance, Beilharz et al.
\cite{beilharz2021}, introduce Marvis, a framework that provides a comprehensive
staging environment for testing distributed IoT applications. Marvis
orchestrates hybrid testbeds, co-simulated domain environments, and a central
network simulation to create a representative operating environment for the
system. However, Marvis does not provide a virtualized energy system, which is
crucial to meet the requirements of our problem statement.

Hagenmeyer et al. \cite{hagenmeyer2016} investigate the interplay of different
forms of energy on various value chains in Energy Lab 2.0. The focus is on
finding novel concepts to stabilize the volatile energy supply of renewables
through the use of storage systems and the application of information and
communication technology tools and algorithms. The smart energy system
simulation and control center is a key element of Energy Lab 2.0 and consists of
three parts: a power-hardware-in-the-loop experimental field, an energy grid
simulation and analysis laboratory, and a control, monitoring, and visualization
center. While this smart energy system simulation is similar to the simulated
Ecovisor in our approach, the control center is the only entity with
software-based control over this energy system. The Ecovisor infrastructure,
however, provides multiple applications with the ability to manage their energy
supply themselves which is an essential factor for our approach.
