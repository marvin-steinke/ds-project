The Software-In-The-Loop (SIL) methodology is frequently utilized during the
creation of intricate systems as a means of conducting repeatable and manageable
component testing.
%TODO connect everything and explain why rw does not solve out problem

Beilharz et al. \cite{beilharz2021} build upon the idea that a staging
environment is necessary to improve IoT applications. They introduce Marvis, a
framework that provides a comprehensive solution for testing distributed IoT
applications. Marvis orchestrates hybrid testbeds, co-simulated domain
environments, and a central network simulation to create a representative
operating environment for the system. The framework provides a comprehensive
solution for testing the complex communication behavior of IoT applications and
is essential for improving their performance. Beilharz et al.'s preliminary
results include an open source prototype and a demonstration of a
Vehicle-to-everything (V2X) communication scenario, showcasing the potential of
Marvis in testing and improving IoT applications.

Hagenmeyer et al. \cite{hagenmeyer2016} investigate the interplay of different
forms of energy on various value chains in Energy Lab 2.0. The focus is on
finding novel concepts to stabilize the volatile energy supply of renewables
through the use of storage systems and the application of information and
communication technology tools and algorithms. The smart energy system
simulation and control center is a key element of Energy Lab 2.0 and consists of
three parts: a power-hardware-in-the-loop experimental field, an energy grid
simulation and analysis laboratory, and a control, monitoring, and visualization
center. The three labs emphasize the importance of big data technologies,
advanced control methods, and reliable, safe, and secure software structures.
Hagenmeyer et al. provide an example of a data processing pipeline to create
power flow simulation models from raw Open Street Map data, statistical
databases, and geodata.
