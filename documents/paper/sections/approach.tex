% TODO units

Figure \ref{fig:system_design} illustrates our general system design approach
that simulates the ecovisor infrastructure and integrates it into the Mosaik
co-simulation environment while enabling SIL capabilities. The present design
can be categorized into two distinct parts, the simulation of the ecovisor, and
its interface to external applications, which are elaborated on in the
following.

\begin{figure}
    \centering
    \includegraphics[width=\linewidth]{figures/system_design}
    \caption{
        General System Design: The ecovisor infrastructure is simulated and
        integrated into the Mosaik co-simulation environment. SIL capabilities
        are enabled via an API Server and a Redis database, connecting the
        simulation environment and real applications in real-time.
    }
    \label{fig:system_design}
\end{figure}

\subsection{Simulation of the Ecovisor}

\begin{figure}
    \removelatexerror
    \begin{algorithm}[H]
        \caption{Virtual Energy System Simulation}
        \label{alg:virtual_energy_system_simulation}
        $rest \gets consumption - solar$\;
        \eIf{$rest \leq 0$} {
            $b\_discharge\_rate \gets 0$\;
        }{
            $b\_discharge\_rate \gets \text{min}($\;
            \Indp
                $b\_max\_discharge,$\;
                $b\_charge\_level \cdot 3600,$\;
                $rest$\;
            \Indm
            $)$\;
            $rest \gets rest - b\_discharge\_rate$\;
        }
        $grid\_power \gets b\_charge\_rate + rest$\;
        $b.delta \gets b\_charge\_rate - b\_discharge\_rate$\;
        $b.step()$\;
        $b\_charge\_level \gets b.charge$\;
        $total\_carbon \gets grid\_carbon \cdot grid\_power$\;
        \vspace{3mm}
    \end{algorithm}
\end{figure}

\subsection{Interface to External Applications}
ToDo: Second itteration on writing

For connecting the Ecovisor-Model to a real workload, we had to expose the API described in \cite{souza2023}. We have tried to implement it as it would probably be implemented in a real cloud environment. %source?
To achive that, we implemented the API of the Ecoviser with a RedisDB as value store between the ecovisor and the api-server. This also ensures atomiticity on the writes of the values. %sourec

The API integration itself cosists of three parts, the API-Server, the RedisDB and the Ecovisor-Redis-Interface
%Picture API
%\begin{figure}
%    \centering
%    \includegraphics[width=\linewidth]{}
%    \caption{Ecovisor API}
%    \label{fig:ecovisor_api}
%\end{figure}



\subsubsection{API-Server}
The API Server exposes the Ecoviser-APi to the "Workloads". It is implemented with the FastAPI Framework and Utilizes Uviorn to handel multiple clients accessing the API.
The Module is started as aan independent thread. In earlyer versions we tried to implement it inside of the ecovisor model, but due to the way uvicorn starts the API-Server,
it would stops the execution of the simulation until the API-Server is stopped and so we decided to design it as a standalone module. This als enables the API-Server to be scaled indipendetn from
the Ecovisor-Model and the RedisDB. This may be helpful in bigger simulations with distributed infrastructur. 


\subsubsection{RedisDB}
The RedisDB is used to interchange the values provided by eiterh ecovisor or the workload application. The database itself is a fast in memory key value store, wich is started as a docker container via the docker python libary %quelle pyton lib
to keep manual managment of the simulation to a minimum.

In general the Database can be exchanged with any other database by adapting the Ecovisor-Redis-Interface in the Ecovisor model. This can be useful when simulation should be integrated 
in a production grade cloud environment like kubernetes or open stack.

%values picture

\subsubsection{Ecovisor-Redis-Interface}


\subsubsection{Dataflow}
%Picture Data-Flow (Values set by Ecovisor, Values set by Workload)
%\begin{figure}
%    \centering
%    \includegraphics[width=\linewidth]{}
%    \caption{Dataflow betwen Ecovisor and API Server}
%    \label{fig:ecovisor_api}
%\end{figure}

