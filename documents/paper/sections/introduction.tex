The surge of cloud platforms in recent years has had a significant impact on
many businesses and individuals, offering access to innovative and valuable
applications that frequently require significant computational resources. With
% AI as prime example, OpenAI as case study?
the increasing demand for more computational power, the cloud platforms have
become an essential part of the digital landscape. These platforms allow for the
storage, processing and management of large amounts of data and computational
resources, which can be used to run applications and services that are beyond
the capabilities of traditional hardware systems.

However, while the growth of cloud platforms has brought many benefits, it has
also raised concerns about their environmental impact. As the demand for
computational resources increases, so does the carbon emissions generated by the
energy consumption of these platforms. The carbon footprint of cloud platforms
has become a major concern, as the energy consumption of these systems is
rapidly increasing, contributing to the growing problem of climate change.
Despite their environmental impact, the growth of cloud platforms shows no signs
of slowing down. To mitigate their impact on the environment, cloud platforms
are now looking for ways to reduce their carbon footprint. It is imperative to
adopt cleaner energy sources for powering data centers, both in the cloud and at
the edge.

Although clean energy offers numerous advantages, it is perceived as being
unreliable due to two key factors. Firstly, the generation of renewable energy
sources such as solar and wind is impacted by environmental variations.
Secondly, the carbon-intensity of grid power also experiences fluctuations as
the grid employs various types of generators with varying carbon emissions to
meet demand. The field of computing possesses a distinct advantage in terms of
reducing its carbon impact through the utilization of cleaner energy sources,
even with their inherent instability. However, current cloud applications are
unable to leverage these benefits to optimize their carbon efficiency. This is
because the energy system obscures the instability of clean energy with a
reliability abstraction, which gives no control or visibility into the energy
supply. As a result, applications cannot adjust their power usage in response to
changes in the availability and carbon-intensity of renewable energy.

Souza et al. proposed a solution to this issue by creating an
\enquote{Ecovisor}, which virtualises the energy system and provides
software-based control over it. This Ecovisor enables each application to manage
the instability of clean energy through software, customized to meet its unique
requirements.

% however, setting up a system with Ecovisor is expensive, time consuming, etc.
% -> simulate Ecovisor
% but we still want real applications in our Ecovisor infra
% -> *only* simulate Ecovisor
% we present our approach that makes this possible
