\IEEEPARstart{T}{he} surge of cloud platforms has had a significant impact on
many businesses and individuals, offering access to innovative and valuable
applications that frequently require significant computational resources. A
notable example for this is represented by OpenAI's chat generative pre-trained
transformer (ChatGPT) cloud platform, which is based on the GPT-3 model
developed by Brown et al. \cite{brown2020} and achieved more than a million
users within the first five days of its launch \cite{brockman2022}. The
increasing demand for more computational power has lead cloud platforms to
become an essential part of the digital landscape \cite{zhang2010}. These
platforms allow for the storage, processing and management of large amounts of
data and computational resources, which can be used to run applications and
services that are beyond the capabilities of traditional hardware systems.

However, while the growth of cloud platforms has brought many benefits, it has
also raised concerns about their environmental impact. As the demand for
computational resources increases, so does the carbon emissions generated by the
energy consumption of these platforms \cite{siddik2021}. Despite their
ecological impact, the growth of cloud platforms shows no signs of slowing down.
According to Gartner Inc. worldwide end-user spending on public cloud services
is forecasted to grow 20.7\% to total \$591.8 billion in 2023
\cite{gartner2022}. To mitigate their impact on the environment, cloud platforms
are now looking for ways to reduce their carbon footprint \cite{google_carbon,
amazon_carbon}. It is imperative to adopt cleaner energy sources for powering
data centers, both in the cloud and at the edge. \medskip

Although clean energy offers numerous advantages, it is perceived as being
unreliable due to two key factors. One, the generation of renewable energy
sources such as solar and wind is affected by environmental changes, and two,
the carbon-intensity of grid power fluctuates as the grid uses different types
of generators with varying carbon emissions to meet demand \cite{scarlat2022}.
The field of computing possesses a distinct advantage in terms of reducing its
carbon impact through the utilization of cleaner energy sources
\cite{murugesan2008}. However, current cloud applications are unable to apply
these benefits to optimize their carbon efficiency. This is because the energy
system obscures the instability of clean energy with a reliability abstraction,
which gives no control or visibility into the energy supply. As a result,
applications cannot adjust their power usage in response to changes in the
availability and carbon-intensity of renewable energy \cite{souza2023}.

Souza et al. \cite{souza2023} proposed a solution to this issue by creating an
\enquote{ecovisor}, which virtualises the energy system and provides
software-based control over it. This ecovisor enables each application to manage
the instability of clean energy through software, customized to meet its unique
requirements. However, even their small-scale prototype is intricately designed
and incorporates several expensive components, such as a DC power supply
equipped with a solar array simulation capability that costs almost \$10,000.

Though the creation of the ecovisor is a promising solution to the issue of
reducing the carbon footprint of cloud platforms, a large scale prototype for
research and development purposes would not only be significantly cost
intensive, but also time consuming. An alternative approach to consider is to
simulate \emph{only} the ecovisor infrastructure with real applications as a
Software-In-The-Loop (SIL) implementation. This would allow for the optimization
of energy usage in response to changes in the availability and carbon-intensity
of renewable energy, without the need for a physical implementation. By
simulating the ecovisor, applications can still manage the instability of clean
energy through software, but at a lower cost and with less time investment.

To this end, we propose the utilization of a co-simulation tool, Mosaik, to
integrate this ecovisor simulation into, and evaluate its impact on the carbon
footprint of cloud platforms. Mosaik is a simulation framework for power
systems, communication networks, and building automation, which can be used to
model and analyze the performance of the ecovisor in real-world applications
\cite{steinbrink2019}. \medskip

In the following sections of this article, we will commence by presenting the
necessary contextual information in Section \ref{sec:background} to ensure
comprehension of our methodology. This includes co-simulation environments with
Software- (SIL) and Hardware-In-The-Loop (HIL) strategies, the Mosaik
co-simulation tool and a closer examination of the ecovisor infrastructure and
its interface to applications. We will then review some related literature with
similar approaches in Section \ref{sec:related_work}. Subsequently, we present
our approach of integrating the ecovisor into Mosaik in Section
\ref{sec:approach}. The approach is divided into the simulation of the ecovisor
itself and the interface to external applications, beyond the scope of the
co-simulation. We evaluate our approach in Section \ref{sec:evaluation} with
exemplary data, representing a realistic scenario that covers all edge cases.
After discussing future research directions related to this article in Section
\ref{sec:future_work}, we conclude this article in Section \ref{sec:conclusion}
by summarizing the main points and contributions.
