\documentclass[a4paper, twoside]{IEEEtran}
\IEEEoverridecommandlockouts
\usepackage[noadjust]{cite}
\usepackage{csquotes}
\usepackage[english]{babel}
\usepackage{cite}
\usepackage{hyperref}
\usepackage{amsmath,amssymb,amsfonts}
\usepackage{graphicx}
\usepackage{textcomp}
\usepackage{xcolor}
\usepackage{array}
\usepackage{tikz}
\usepackage{etoolbox}
\usepackage{url}
\usepackage[linesnumbered]{algorithm2e}
\usepackage{enumitem}
\usepackage{pgfplots}
\pgfplotsset{compat=newest}

% make algorithm2e work with IEEEtran
\makeatletter
\newcommand{\removelatexerror}{\let\@latex@error\@gobble}
\makeatother

% adjust algorithm2e to IEEEtran style
\SetAlCapNameFnt{\footnotesize}
\SetAlCapFnt{\footnotesize}

% remove semicolons in algorithm env
\BeforeBeginEnvironment{algorithm}{\DontPrintSemicolon}

% group citations
\renewcommand{\citepunct}{,\penalty\citepunctpenalty\,}
\renewcommand{\citedash}{--}

% make urls work in biblatex
\def\UrlBreaks{\do\/\do-}
\apptocmd{\sloppy}{\hbadness 10000\relax}{}{}
\def\BibTeX{{\rm B\kern-.05em{\sc i\kern-.025em b}\kern-.08em
T\kern-.1667em\lower.7ex\hbox{E}\kern-.125emX}}

% cant fix underful hboxes with that many underscores in variable names
\hbadness=10000

\begin{document}

\title{Integrating Ecovisor into Mosaik Co-Simulation}
\markboth{Distributed and Operating Systems, TU Berlin, March 2023}{Steinke \&
Nickel: Integrating Ecovisor into Mosaik Co-Simulation}
\author{Marvin Steinke and Henrik Nickel\\\textit{Technische Universität Berlin}}

\maketitle

\begin{abstract}

    To reduce emissions, cloud platforms must increasingly rely on renewable
    energy sources such as solar and wind power. Nevertheless, the issue of
    volatility associated with these sources presents a significant challenge,
    since current energy systems conceal such unreliability in hardware. Souza
    et al. have devised a solution to this issue by creating an
    \enquote{ecovisor}. This system virtualizes the energy infrastructure and
    allows for software-defined control to be accessible by applications.
    Setting up the ecovisor to develop carbon-aware applications, however, can
    be costly and time consuming. To address this problem, we simulated the
    ecovisor and its virtual energy system and integrated in into a Mosaik
    co-simulation. With an API server and a Redis database we are enabling
    Software- (SIL) and Hardware-In-The-Loop (HIL) capabilities. To evaluate our
    approach, we created test cases using recorded solar and carbon data to
    demonstrate the accuracy of the ecovisor model's implementation and its
    ability to transfer data correctly between the model and the API server.

\end{abstract}

\section{Introduction}
\label{sec:introduction}
\begin{frame}{Carbon-Aware Computing}
    \begin{columns}
        \column{.5\textwidth}
        \begin{itemize}
            \item data centers' energy consumption is a concern for carbon emissions
            \item carbon- and renewable-aware computing can optimize efficiency
            \item \textbf{virtual energy systems} and \textbf{software defined
                control} can be used to achieve this
        \end{itemize}
        \column{.5\textwidth}
        \begin{figure}
            \centering
            \includegraphics[height=.7\textwidth]{../../tree_pc}
            \caption{DALL-E 2 \enquote{a tree, growing out of an old computer}}
            \label{fig:tree_pc}
        \end{figure}
    \end{columns}
\end{frame}


\section{Background}
\label{sec:background}
\subsection{Co-Simulations and SIL / HIL}

Co-simulations are used to model and analyze complex systems with multiple
interacting components, each of which may have different properties and
behaviors. They involve combining simulation models from different domains, such
as control, power, and thermal management, to create a unified model that
accurately represents the behavior of the overall system.

One of the main advantages of co-simulations over regular simulations is their
ability to capture the interactions between different components of the system.
Regular simulations often make simplifying assumptions that can lead to
inaccurate results. Co-simulations, on the other hand, can account for the
interactions between components and provide a more accurate representation of
the system's behavior. This makes co-simulations particularly useful for
designing and optimizing complex systems like datacenters. The virtual
environment can save time and resources, reduce the risk of failure, and lead to
more efficient and sustainable datacenters \cite{vogt2018}.

Co-simulations can be further enhanced by incorporating the SIL and HIL
methodology. This approach involves integrating real-world components, such as
software and hardware, into the simulation environment to better reflect the
actual system behavior. The integration of real components allows for a more
accurate representation of the system's behavior and can also identify potential
issues that may arise in real-world scenarios \cite{kelemenova2013}.

\subsection{Mosaik}

Mosaik is an open-source co-simulation tool that allows for the integration of
different simulation models from various domains into a unified simulation
environment. Mosaik provides a Python-based framework for developing and
executing co-simulations, enabling the creation of complex simulations with
interacting components. It supports the development of co-simulation scenarios
by providing an API for defining simulation models, connecting them to form a
simulation network, and specifying simulation scenarios. The tool also provides
various visualization tools and data analysis capabilities to analyze the
results of the simulation. Furthermore, Mosaik provides a library of
pre-existing simulation models that can be used to build custom simulations.

\subsection[Ecovisor]{
    Ecovisor \footnote{The information presented in this section is a summary of
    Section 3 and 3.1 from Souza et al.'s work \cite{souza2023}, with some
    modifications made for clarity and conciseness.}
}

Figure \ref{fig:ecovisor_design} shows an overview of the ecovisor's design
which manages resources and energy using containers or virtual machines as the
basic unit. An instance-level API is chosen to align with existing cloud APIs
and to support higher-level cluster or cloud-level APIs. The ecovisor extends an
existing orchestration platform that provides basic container or VM management
and monitoring functions. Container Orchestration Platforms (COPs) are used to
manage resources and applications. COPs provide virtual clusters composed of
multiple containers with specified resource allocations that can grow or shrink
over time. COPs include a scheduling policy that determines resource allocation
under constraints, and COPs are resilient to resource revocations. This
resiliency is useful for designing carbon-efficient applications as low-carbon
energy may cause power shortages that also manifest as resource revocations.

The virtual energy system includes a virtual grid connection, a virtual battery,
and a virtual solar array. The system provides getters and setters methods for
monitoring and controlling the virtual power supply and demand, including
per-container power caps, battery charging, and discharging rates, as shown in
Table \ref{table:ecovisor_api}. The system uses virtual solar power first to
meet demand and charges the virtual battery with any excess solar power. When
there is not enough solar power, the virtual energy system uses grid power to
make up the difference, while accounting for carbon emissions and power usage
over discrete time intervals. The ecovisor system provides a uniform centralized
interface for accessing energy-related information and historical data.

\begin{figure}
    \centering
    \includegraphics[width=\linewidth]{figures/ecovisor_design}
    \caption{Ecovisor Design (Souza et al.) \cite{souza2023}}
    \label{fig:ecovisor_design}
\end{figure}

\begin{table*}
    \centering
    \caption{Ecovisor's API (Souza et al.) \cite{souza2023}}
    \label{table:ecovisor_api}
    \begin{tabular}{||l|c|c|c|c||}
        \hline
        \textbf{Function Name} & \textbf{Type} & \textbf{Input} & \textbf{Return
        Value} & \textbf{Description} \\
        \hline\hline
        \texttt{set\_container\_powercap()} & Setter & ContainerID, kW & N/A & Set
        a container's power cap \\
        \hline
        \texttt{set\_battery\_charge\_rate()} & Setter & kW & N/A & Set battery charge rate until full \\
        \hline
        \texttt{set\_battery\_max\_discharge()} & Setter & kW & N/A & Set max battery discharge rate \\
        \hline\hline\hline
        \texttt{get\_solar\_power()} & Getter & N/A & kW & Get virtual solar power output \\
        \hline
        \texttt{get\_grid\_power()} & Getter & N/A & kW & Get virtual grid power usage \\
        \hline
        \texttt{get\_grid\_carbon()} & Getter & N/A &
        g\,$\cdot$\,CO\textsubscript{2}/kW & Get virtual grid power usage \\
        \hline
        \texttt{get\_battery\_discharge\_rate()} & Getter & N/A & kW & Get current rate of battery discharge \\
        \hline
        \texttt{get\_battery\_charge\_level()} & Getter & N/A & kWh & Get energy
        stored in virtual battery \\
        \hline
        \texttt{get\_container\_powercap()} & Getter & ContainerID & kW & Get a
        container's power cap \\
        \hline
        \texttt{get\_container\_power()} & Getter & ContainerID & kW & Get a
        container's power usage \\
        \hline\hline\hline
        \texttt{tick()} & Notification & N/A & N/A & Invoked by ecovisor every
        $\Delta t$ \\
        \hline
    \end{tabular}
\end{table*}


\section{Related Work}
\label{sec:related_work}
Beilharz et al. \cite{beilharz2021} developed Marvis, a testing framework for
distributed IoT applications. Marvis provides realistic staging environments for
IoT applications by using a combination of virtual and physical nodes that
communicate via virtual or physical networks. The domain environments in Marvis
are modeled after the real-world environments in which the IoT applications will
be used, and Marvis integrates simulators, emulators, and hardware testbeds to
model the expected behavior of these environments. The experiments are executed
in wall clock time to evaluate the timing effects of physical components.
Beilharz et al. use hybrid co-simulation to integrate simulators into the
testing environment, which involves the synchronized execution of systems with
both discrete-event and continuous time behavior.


\section{Approach}
\label{sec:approach}
\begin{frame}{Requirements}
    \begin{columns}
        \column{.5\textwidth}
        \begin{itemize}
            \item original Ecovisor design is abstracted to a model with
                full functionality
            \item Ecovisor model is executed within Mosaik
            \item simulated consumers can access API via Mosaik's interface
            \item real consumer \emph{outside} the simulation can access API
                in real time
        \end{itemize}
    \end{columns}
\end{frame}

\begin{frame}{Design}
    \begin{figure}
    \centering
    \includegraphics[height=.68\textheight]{../../system_design}
    \caption{System design}
    \label{fig:system_design}
    \end{figure}
\end{frame}


\section{Evaluation}
\label{sec:evaluation}
To assess the effectiveness of our system, we conducted a two-part evaluation
using open-source real-world data. We designed two example cases to demonstrate
the simulation capabilities of Ecovisor and showcase how the implemented API can
be used by consumers. To improve the clarity of the data presented in our
examples, we scaled down the Photo Voltaic (PV) data to approximately 30\% of
its original values. This allowed us to better showcase the API usage and
plotted graphs. Scaling down the input data did not impact the simulation
itself, as we could simply assume that less solar power was available due to
weather or limitations of the solar power generators.

In the first part of our evaluation, we demonstrate the general functionality of
the Ecovisor model and showcased how the output changed with varying input of
the energy mix. In the second example, we highlighted a missing functionality
of the API by demonstrating how a workload application can set the
\texttt{battery\_charge\_rate} and \texttt{battery\_max\_discharge}.

To enhance the comprehensibility of our evaluation, we included two plots
visualizing the output data from the simulation. One plot was included for each
example, helping to clearly illustrate the results of our evaluation.

%To evaluate the functionality of our system, we designed two example-cases to show, how the simulation of the Ecovisor works and how the implemented API is used by a consumer. For this we used open source real world data.
%To fit the data into our examples, we scaled the Photo Voltaic (PV) Data down to ~30\% of its original values. This enhances the visibility fo the API usage and the data in the plotted graphs. Scaling down the input data does not affect the simulation itself, since we can just assume, that less solar power is available due to weather or limitations of the solar power generators.\\
%We divided our showcases into two parts, the first one showing the general functionality of the Ecovisor model and how the output changes with different changing input of the energy mix.\\
%The second example shows the missing functionality of the API, which is used by a workload application to set the \texttt{battery\_charge\_rate} and the \texttt{battery\_max\_discharge}.

\subsection{Ecovisor-Model with changing energy input}

\begin{figure*}
    \centering
    \begin{tikzpicture}
        \begin{axis}[
                xlabel={Time in s},
                ylabel={Energy in kWs},
                ymajorgrids=true,
                grid style=dashed,
                legend pos=outer north east,
            ]

            \addplot[color=red, mark=dot]
            table [x=time, y=consumption, col sep=comma]
                {figures/scenario_b.csv};
            \addlegendentry{Consumption}

            \addplot[color=yellow, mark=dot]
            table [x=time, y=solar_power, col sep=comma]
                {figures/scenario_b.csv};
            \addlegendentry{PV Power}

            \addplot[color=gray, mark=dot]
            table [x=time, y=grid_power, col sep=comma]
                {figures/scenario_b.csv};
            \addlegendentry{Grid Power}

            \addplot[color=blue,mark=dot]
            table[x=time, y=battery_charge_level, col sep=comma]
                {figures/scenario_b.csv};
            \addlegendentry{Battery Charge Level}

        \end{axis}
    \end{tikzpicture}
    \caption{Showcase changing Energy Mix}
    \label{fig:example_case_a}
\end{figure*}



The first example showcased the general functionality of the Ecovisor model and
how the output is affected by changing availability of different energy sources.
For this simulation, we utilized an example carbon data to determine the amount
of carbon emitted when using grid energy and a solar energy input file, which
was scaled down to one-third of its original value. Starting from second 200, we
further scaled it down until the output was under the total consumption of the
workload application. In addition, we set the \texttt{battery capacity} to 0.3
KW/h and the \texttt{battery\_charge\_level} to 0.15 KW/h.

In the plot \ref{fig:example_case_a}, we can observe that the \texttt{Grid
Power} output is opposite to the PV Power output until second 200. This is
because when enough PV energy is available, the surplus energy is fed into the
energy grid, resulting in negative carbon emission. During the first 200
seconds, the workload application's consumption is stable at 500 KWs and doesn't
change throughout the simulation. Although this isn't ideal in a real-world
scenario, it allowed us to more easily demonstrate the functionality of the
model.

In the first third of the simulation, until second 100, we can see that there is
enough PV energy to run the workload and feed some extra energy into the grid.
However, the Battery Charge Level isn't high enough to load the battery
consistently. Around second 100, the PV Energy level rises, and the battery
starts charging until second 200. At second 200, the PV Energy level drops, and
the workload application uses the energy in the battery and from the available
PV energy to meet its demands. Around second 350, no PV Energy is available, and
the workload only runs on grid power until the end of the simulation. The
battery doesn't charge since it only charges with PV energy.

This example demonstrates how the Ecovisor model behaves with different
available energy sources. It charges the battery when enough PV energy is
available and uses all other available energy before utilizing grid energy to
reduce carbon emission.


\subsection{Additional API Functionality}

\begin{figure*}
    \centering
    \begin{tikzpicture}
        \begin{axis}[
                xlabel={Time in s},
                ylabel={Energy in kWs},
                ymajorgrids=true,
                grid style=dashed,
                legend pos=outer north east,
            ]

            \addplot[color=red, mark=dot]
            table [x=time, y=consumption, col sep=comma]
                {figures/scenario_a.csv};
            \addlegendentry{Consumption}

            \addplot[color=yellow, mark=dot]
            table [x=time, y=solar_power, col sep=comma]
                {figures/scenario_a.csv};
            \addlegendentry{PV Power}

            \addplot[color=gray, mark=dot]
            table [x=time, y=grid_power, col sep=comma]
                {figures/scenario_a.csv}; \addlegendentry{Grid Power}

            \addplot[color=blue,mark=dot]
            table[x=time, y=battery_charge_level, col sep=comma]
                {figures/scenario_a.csv};
            \addlegendentry{Battery Charge Level}

        \end{axis}
    \end{tikzpicture}
    \caption{Showcase additional API Functionality}
    \label{fig:example_case_b}
\end{figure*}

The second showcase, illustrated in Figure \ref{fig:example_case_b}, employs the
same PV and Carbon data as in the first showcase, resulting in the same
correlation between PV Energy and Grid Energy. The objective of this showcase is
to demonstrate the usage of the API endpoints for setting the
\texttt{container\_powercap}, the \texttt{battery\_charge\_rate}, and the
\texttt{battery\_max\_discharge}.

While the \texttt{container\_powercap} was also utilized in the first showcase,
the consumption set in the beginning remained constant. In this showcase, we
developed a Python script that emulates a workload application by sending API
requests to adjust the \texttt{container\_powercap} around second 45 from 500
KWs to 250 KWs, which is clearly discernible in the plot. This capability
enables a workload application to dynamically adjust its power consumption while
scaling up or down.

Furthermore, the \texttt{battery\_charge\_rate} and
\texttt{battery\_max\_discharge} are utilized to operate the battery. The
\texttt{battery\_charge\_rate} determines how much of the available PV Energy is
used to charge the battery, while the \texttt{battery\_max\_discharge} sets the
threshold of energy that will not be used. These parameters can be adjusted to
either extend the battery's lifespan or to implement different energy usage
profiles. We changed both values by sending API requests with the same Python
script used to set the \texttt{container\_powercap}. At around second 75, we can
observe that after setting both battery control values to 10 KWs, the battery
begins to charge even when the available energy does not change.

These two examples demonstrate the functionality of the Ecovisor-Model and its
API. They show how the model can be used to simulate an Ecovisor for a workload
application, allowing researchers and engineers to test different energy usage
profiles and settings.


\section{Future Work}
\label{sec:future_work}
While our approach focused on a single simulation of the ecovisor, our use of
Mosaik demonstrates its effectiveness for large-scale smart grid simulations. In
future research, we suggest interconnecting multiple ecovisor systems to share
resources and further reduce carbon emissions. This network could be distributed
across different geographic regions, as carbon intensity varies depending on
location. By incorporating carbon information services like Electricity Maps
\cite{electricity_maps}, this could enable carbon-efficiency optimizations such
as Let’s Wait Awhile \cite{wiesner2021} or Cucumber \cite{wiesner2022} from
Wiesner et al. This would enhance the potential for carbon reduction at a larger
scale. This work also does not focus on monitoring and limiting the power
consumption of applications in the loop, due to the specific requirements of the
implementation. Therefore, a potential future direction could involve
incorporating an interface that enables various power measurement and control
techniques for different application hosts, such as containers, virtual
machines, or physical hardware systems.


\section{Conclusion}
\label{sec:conclusion}
\begin{frame}{Conclusion}
    \begin{itemize}
        \item \textbf{Ecovisor} -- handle clean energy's unreliability in
            software
        \item \textbf{Mosaik} -- combine multiple simulations
        \item[\arrow] \textbf{Approach} -- real-time workload modeling with
            carbon control
        \item[\arrow] \textbf{TODO} -- enable carbon-efficiency optimizations
            with geo-distributed Ecovisors
    \end{itemize}
\end{frame}


\bibliographystyle{IEEEtran}
\bibliography{refs}

\end{document}
